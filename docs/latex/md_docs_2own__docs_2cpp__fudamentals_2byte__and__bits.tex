\chapter{Bytes and Bits}
\hypertarget{md_docs_2own__docs_2cpp__fudamentals_2byte__and__bits}{}\label{md_docs_2own__docs_2cpp__fudamentals_2byte__and__bits}\index{Bytes and Bits@{Bytes and Bits}}
\label{md_docs_2own__docs_2cpp__fudamentals_2byte__and__bits_autotoc_md40}%
\Hypertarget{md_docs_2own__docs_2cpp__fudamentals_2byte__and__bits_autotoc_md40}%
 A byte is a unit of digital information that most commonly consists of eight bits. A bit is the most basic unit of information in computing and digital communications, which can have only one of two values, often represented as 0 or 1.

Here is a brief description of the C++ primitive types, including their binary and hexadecimal representations for the value {\ttfamily 7} as an example\+:


\begin{DoxyEnumerate}
\item {\ttfamily bool}\+: Boolean type, values can be {\ttfamily true} or {\ttfamily false}. Binary and hexadecimal representation isn\textquotesingle{}t typically used with booleans.
\item {\ttfamily char}\+: Character type. It\textquotesingle{}s exactly one byte in C++. Binary\+: {\ttfamily 00000111}, Hexadecimal\+: {\ttfamily 07}
\item {\ttfamily int}\+: Integer type. Its size can vary, but it\textquotesingle{}s at least 2 bytes in C++. Binary\+: {\ttfamily 00000000 00000111} for 2 bytes, Hexadecimal\+: {\ttfamily 0007} for 2 bytes.
\item {\ttfamily float} and {\ttfamily double}\+: Floating point types. The size of {\ttfamily float} is 4 bytes, and the size of {\ttfamily double} is 8 bytes. These types use a complex representation that\textquotesingle{}s not easily readable in binary or hexadecimal.
\item {\ttfamily void}\+: The type specifier {\ttfamily void} indicates that no value is available.
\item {\ttfamily wchar\+\_\+t}\+: A wide character type. Its size is compiler-\/specific, with binary and hexadecimal representations depending on its width.
\end{DoxyEnumerate}

Note that the sizes of the types can vary between different systems and compilers. You can always use the {\ttfamily sizeof} operator in C++ to get the size of a specific type on your system.

Also note that the binary and hexadecimal representations depend on the specific value of the variable. The examples above are for the value {\ttfamily 7}.

The C++ standard library also includes several integer types with specified widths, such as {\ttfamily int16\+\_\+t}, {\ttfamily uint32\+\_\+t}, {\ttfamily int64\+\_\+t}, etc. These types are useful when you need integers of a specific size.

Keep in mind, bit-\/level representations for floating-\/point types ({\ttfamily float}, {\ttfamily double}) are complex due to the way floating-\/point values are stored, which involves a sign, a mantissa, and an exponent. The binary and hexadecimal representations of these types are not straightforward like they are for integers. 